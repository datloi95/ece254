\input{configuration}

\title{Lecture 15 --- Deadlock }

\author{Jeff Zarnett \\ \small \texttt{jzarnett@uwaterloo.ca}}
\institute{Department of Electrical and Computer Engineering \\
  University of Waterloo}
\date{\today}


\begin{document}

\begin{frame}
  \titlepage

 \end{frame}

\begin{frame}
\frametitle{Deadlock}

An informal definition of deadlock: all processes being ``stuck''. 

A more formal definition: ``the \textit{permanent} blocking of a set of processes that either compete for system resources or communicate with each other''.

It may be possible for all processes to be stuck temporarily, because one is waiting for some event (e.g., a read from disk).

This situation will resolve itself and is not deadlock. 

\end{frame}


\begin{frame}
\frametitle{Deadlock}
 A set of processes is truly deadlocked when each process in the set is blocked on some event that can only be triggered by another blocked process in the set. 
 
 In this case it is permanent, because none of the events can take place.

A deadlock involves some conflicting needs for resources by two or more processes. 

\end{frame}

\begin{frame}
\frametitle{A Traffic Deadlock}

\begin{center}
\includegraphics[width=\textwidth]{images/car-deadlock.png}
\end{center}


\end{frame}

\begin{frame}
\frametitle{A Traffic Deadlock}

Whichever vehicle arrives at the stop sign first has the right of way. 

If two vehicles arrive at the same time, a vehicle yields the right of way to the vehicle on its right.

Okay, as long as 3 or fewer vehicles come to the stop sign at the same time.

 If all four vehicles arrive at the same time, we have the potential for a problem. 
\end{frame}

\begin{frame}
\frametitle{A Traffic Deadlock}

It is not a deadlock yet, because none of the processes are stuck yet.

If all the drivers believe they should go first, we get the situation on the right, and we actually do have deadlock. 

This is very much like the dining philosophers problem; deadlock occurs if everyone tries to do the same thing at the same time.

\end{frame}

\begin{frame}
\frametitle{Deadlocks}

Of course, for deadlock to occur, we do not have to have symmetric processes trying to do the same thing at the same time. 

Given two semaphores, \texttt{a} and \texttt{b}, and two processes, we can have the following code that will sometimes, but not always lead to deadlock. 

If thread $P$ locks \texttt{a} and then there is a process switch, and \texttt{b} is locked by $Q$, both threads will be stuck. 

Each has one resource the other needs, but they are both blocked.


\end{frame}

\begin{frame}[fragile]
\frametitle{Deadlock of $P$ and $Q$}
It's not always this easy to see:


\begin{multicols}{2}
\textbf{Thread P}
  \begin{verbatim}
	 1. wait( a ) 
	 2. wait( b )
	 3. [critical section]
	 4. signal( a )
	 5. signal( b )
  \end{verbatim}
\columnbreak
\textbf{Thread Q}
  \begin{verbatim}
	 1. wait( b ) 
	 2. wait( a )
	 3. [critical section]
	 4. signal( b )
	 5. signal( a )
  \end{verbatim}
\end{multicols}

\end{frame}

\begin{frame}
\frametitle{Joint Progress Diagram}

We can represent the code with a \alert{joint progress diagram}:

\begin{center}
\includegraphics[width=0.7\textwidth]{images/joint-progress}
\end{center}

There are six execution paths.

\end{frame}

\begin{frame}
\frametitle{The Fatal Region}

The area of the diagram where it says ``Deadlock Inevitable'' is the \alert{fatal region}. 

The existence of a fatal region means that deadlock is inevitable.

Any one particular execution of the program may or may not result in deadlock.

For a situation of $n$ processes sharing resources, an $n$-dimensional diagram would be needed.

If $n \geq 4$, I at least can't represent it visually on a 2D surface.\\
\quad And it has nothing to do with my poor art skills.


\end{frame}

\begin{frame}
\frametitle{Reusable and Consumable Resources}
We can generally classify a resource as either \alert{reusable} or \alert{consumable}. 

A reusable resource can be used by one process at a time, and is not depleted.

A process may lock the resource, make use of it, then release it such that other processes may acquire it. 

Processors, memory, files, semaphores are all examples of reusable resources.

\end{frame}

\begin{frame}
\frametitle{Reusable and Consumable Resources}

A consumable resource is one that is created and destroyed upon consumption. 

If the user presses the ``Z'' key on the keyboard, this generates an interrupt and produces the ``Z'' character in a buffer. 

A process that takes input will then consume that character (e.g., it goes into the \texttt{vi} editor window) and it is unavailable to other processes.

\end{frame}

\begin{frame}
\frametitle{Conditions for Deadlock}

When a disaster happens, it is typically a result of a chain of things going wrong. 

If any one of those things did not happen, the disaster would be averted. 

This is referred to as ``breaking the chain''.

\end{frame}

\begin{frame}
\frametitle{Conditions for Deadlock}
There are four conditions for deadlock:

\begin{enumerate}
	\item \textbf{Mutual Exclusion}
	\item \textbf{Hold-and-Wait}
	\item \textbf{No Preemption}
	\item \textbf{Circular-Wait}
\end{enumerate}


If the first three conditions are true, deadlock is possible, but deadlock will only happen if the fourth condition is fulfilled. 


\end{frame}

\begin{frame}
\frametitle{Resource Allocation Graph}

\begin{center}
\includegraphics[width=0.50\textwidth]{images/rag1.png}
\end{center}

\end{frame}

\begin{frame}
\frametitle{Resource Allocation Graph}

If there are no cycles in the graph, no process in the system is deadlocked. 

If a cycle exists, then some process \textit{may} be deadlocked:

\begin{center}
\includegraphics[width=0.40\textwidth]{images/rag2.png}
\end{center}

\end{frame}

\begin{frame}
\frametitle{Resource Allocation Graph}

But the presence of a cycle is not necessarily certain that there is a deadlock.

\begin{center}
\includegraphics[width=0.40\textwidth]{images/rag3.png}
\end{center}


\end{frame}

\begin{frame}
\frametitle{Dealing with Deadlock}

There are four basic approaches to dealing with deadlock.

\begin{enumerate}
	\item Ignore it.
	\item Deadlock prevention.
	\item Deadlock avoidance.
	\item Deadlock detection.
\end{enumerate}

\end{frame}

\begin{frame}
\frametitle{Ignoring Deadlock}

This option is certainly convenient for operating system designers!

We simply pretend that deadlock can never happen. 

Or if it does happen, it is the application developers' fault. 

Maybe unrealistic.

But, Microsoft Windows takes this approach.

\end{frame}

\begin{frame}
\frametitle{Deadlock Prevention}

This approach is a way of preventing a deadlock from being possible. 

The first three conditions for deadlock (mutual exclusion, hold and wait, and no preemption) are all necessary for deadlock to be possible. 

If we eliminate one of these three pillars, deadlock is not possible and it is prevented from happening.

\end{frame}

\begin{frame}
\frametitle{Knocking Down: Mutual Exclusion}

Mutual Exclusion cannot, generally speaking, be disallowed. 

The purpose of  mutual exclusion is: prevent errors like inconsistent state/crashes. 

Getting rid of mutual exclusion to rule out the possibility of deadlock is a cure that is worse than the disease.

It is therefore not acceptable as a solution.


\end{frame}

\begin{frame}
\frametitle{Knocking Down Hold-and-Wait}
 To prevent the hold-and-wait condition, we must guarantee that when a process requests a resource, it does not have any other resource. 
 
This does not mean that things can be requested only one at a time.

One plausible solution: request all resources at the beginning of the program.

If the program needs $R_{1}$, $R_{2}$, and $R_{3}$ at some point, all three must be requested right at the beginning and held throughout the program.

\end{frame}

\begin{frame}
\frametitle{Knocking Down Hold-and-Wait}

A process has to know in advance all of the resources that it will need. 

Remember that a file is a resource. 

A simple text editor can be used by the user to open an arbitrary file. 

How do we know in advance which will be requested?

\end{frame}

\begin{frame}
\frametitle{Knocking Down Hold-and-Wait}

This also has performance implications: a process cannot start until it has all the resources it will ever need, even if it will not need them until much later. 

Thus, processes might spend a lot of time waiting before starting.

In theory, a process might never start if one or more of the resources it needs is always in use (so this is vulnerable to starvation).


\end{frame}

\begin{frame}
\frametitle{Knocking Down Hold-and-Wait}

What if a process must release all its currently-held resources before it can get any new ones? 

The process has $R_{1}$ and $R_{2}$ and wants to get $R_{3}$.\\
\quad First release $R_{1}$ and $R_{2}$ before it can request all three. 

A resource that cannot be easily released is memory.\\
\quad Released memory may be collected and reassigned by the operating system. 

Therefore we cannot release all resources.

We cannot rule out deadlock; we can only make it less likely to occur.

\end{frame}

\begin{frame}
\frametitle{Two-Phase Locking}

Another idea that might work is \alert{two-phase locking}. 

A process attempts to lock a group of resources at once. 

If it does not get everything it needs, it releases the locks it got and tries again. 

Thus a process does not wait while holding resources. 

If a philosopher picks up a chopstick but is unable to acquire a second, she puts down the chopstick she has picked up and tries again.

\end{frame}

\begin{frame}
\frametitle{Two-Phase Locking}

Two phase locking is not applicable to our current model for semaphores.

There is no way to know the value of the semaphore and the operating system will block a process on a \texttt{wait} if some other thread is in the critical region. 

After the process is blocked on the semaphore, a second process will run, and the first process does not get the opportunity to release the resources it holds. 

There are systems that have nonblocking requests \& mutual exclusion.

Then the program is responsible for checking if any of the requests returned \texttt{false} and releasing any resources where the request returned \texttt{true}.


\end{frame}


\begin{frame}
\frametitle{Knocking Down No Preemption}

Preemption: forcible removal of resources from a process.

Suppose a process $P_{1}$ holds $R_{1}$ and $R_{2}$ and wants to get $R_{3}$, but $R_{3}$ is unavailable. 

$P_{1}$ will be blocked by the operating system. 

If $P_{2}$ requests $R_{1}$ and $R_{2}$, the resources $R_{1}$ and $R_{2}$ are taken away from $P_{1}$.

The resources are added to the list of things that $P_{1}$ is waiting for.

In the meantime, $P_{2}$ can use them and continue. 

$P_{1}$ will be unblocked when all three resources are once again available for it.

\end{frame}


\begin{frame}
\frametitle{Allowing Preemption}

For preemption to work, however, the resource must be a resource of a type where the state can be saved and restored (e.g., the CPU with its registers). 

This is not applicable to all resources; if a printer is in use by $P_{1}$ it cannot be preempted and given to $P_{2}$, otherwise the printout will be a jumble. 

Thus, preemption is also not sufficient to prevent deadlock from ever transpiring, it once again only makes it less likely.

\end{frame}


\end{document}

